对于一个 \textbf{平等} 游戏,可以为每个状态定义一个 SG 函数。

一个状态的 SG 函数等于所有它能一步到达的状态的 SG 函数的 $\text{mex}$,也就是最小的没有出现过的自然数。

那么所有先手必败态的SG函数为$0$,先手必胜态的SG函数非$0$。

如果有一个游戏,它由若干个独立的子游戏组成,且每次行动时 \textbf{只能选一个} 子游戏进行操作,则这个游戏的 SG 函数就是所有子游戏的SG函数的异或和。比如最经典的Nim游戏,每次只能选一堆取若干个石子。

同时操作多个子游戏的结论参见 \ref{classicgame}.\nameref{classicgame} (\pageref{classicgame}页)。
