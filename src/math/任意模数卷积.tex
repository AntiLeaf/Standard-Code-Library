三模数 NTT 和直接拆系数 FFT 都太慢了,不要用。

MTT 的原理就是拆系数 FFT,只不过优化了做变换的次数。

考虑要对 $A(x), B(x)$ 两个多项式做 DFT,可以构造两个复多项式

$$ P(x) = A(x) + iB(x) \quad Q(x) = A(x) - iB(x) $$

只需要 DFT 一个,另一个 DFT 实际上就是前者反转再取共轭,再利用

$$ A(x) = \frac {P(x) + Q(x)} 2 \quad B(x) = \frac {P(x) - Q(x)} {2i} $$

即可还原出 $A(x), B(x)$。

IDFT 的道理更简单,如果要对 $A(x)$ 和 $B(x)$ 做 IDFT,只需要对 $A(x) + i B(x)$ 做 IDFT 即可,因为 IDFT 的结果必定为实数,所以结果的实部和虚部就分别是 $A(x)$ 和 $B(x)$。

\textbf{实际上任何同时对两个实序列进行 DFT,或者同时对结果为实序列的 DFT 进行逆变换时都可以按照上面的方法优化,可以减少一半的 DFT 次数。}

\inputminted{cpp}{../src/math/MTT.cpp}
