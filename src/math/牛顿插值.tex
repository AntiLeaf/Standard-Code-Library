牛顿插值的原理是\textbf{二项式反演}.

二项式反演: 

$$ \begin{aligned}
f(n)=\sum_{k=0}^n{n\choose k}g(k)\;\Leftrightarrow\;g(n)=\sum_{k=0}^n\left(-1\right)^{n-k}{n\choose k}f(k)
\end{aligned} $$

可以用$e^x$和$e^{-x}$的麦克劳林展开式证明.

套用二项式反演的结论即可得到牛顿插值: 

$$ \begin{aligned} f(n)=\sum_{i=0}^{k}{n\choose i}r_i \end{aligned} $$
$$ \begin{aligned} r_i=\sum_{j=0}^i(-1)^{i-j}{i\choose j}f(j) \end{aligned} $$

其中$k$表示$f(n)$的最高次项系数.

实现时可以用$k$次差分替代右边的式子:

\begin{minted}{cpp}
for (int i = 0; i <= k; i++)
    r[i] = f(i);
for (int j = 0; j < k; j++)
    for (int i = k; i > j; i--)
        r[i] -= r[i - 1];
\end{minted}

注意到预处理$r_i$的式子满足卷积形式,必要时可以用FFT优化至$O(k\log k)$预处理.