牛顿插值的原理是 \textbf{二项式反演}。

\paragraph{二项式反演}

$$ f(n) = \sum_{k = 0} ^ n {n \choose k} g(k) \; \iff \; g(n) = \sum_{k = 0} ^ n \left( -1 \right) ^ {n - k} {n \choose k} f(k) $$

可以用 $e^x$ 和 $e^{-x}$ 的麦克劳林展开式证明。

套用二项式反演的结论即可得到牛顿插值:

$$ f(n) = \sum_{i = 0} ^ k {n \choose i} r_i , \; \text{where} \; r_i = \sum_{j = 0} ^ i (-1) ^ {i - j} {i \choose j} f(j) $$

其中 $k$ 表示 $f(n)$ 的最高次项系数。

实现时右边的式子等价于 $k$ 次差分:

\inputminted{cpp}{../src/math/牛顿插值.cpp}

注意到预处理 $r_i$ 的式子满足卷积形式,必要时可以用 FFT 优化至 $O(k\log k)$。
