如果要求出一个次数为$k$的递推式, 则输入的数列需要至少有$2k$项.

返回的内容满足$\sum_{j = 0} ^ {m - 1} a_{i - j} c_j = 0$, 并且$c_0 = 1$. 称为最小递推式.

如果不加最后的处理的话, 代码返回的结果会变成$a_i = \sum_{j = 0} ^ {m - 1} c_{j - 1} a_{i - j}$, 有时候这样会方便接着跑递推, 需要的话就删掉最后的处理.

(实际上Berlekamp-Massey是对每个前缀都求出了递推式, 但一般没啥用.)

\inputminted{cpp}{../src/math/Berlekamp-Massey.cpp}

如果要求向量序列的递推式, 就把每位乘一个随机权值(或者说是乘一个随机行向量$v^T$)变成求数列递推式即可.

如果是矩阵序列的话就随机一个行向量$u^T$和列向量$v$, 然后把矩阵变成$u^T A v$的数列就行了.

\label{BerlekampMasseyApplication}

\subsubsection{优化矩阵快速幂DP}

	如果$f_i$是一个向量, 并且转移是一个矩阵, 那显然$\{f_i\}$是一个线性递推序列.

	假设$f_i$有$n$维, 先暴力求出$f_{0\textasciitilde 2n - 1}$, 然后跑Berlekamp-Massey, 最后调用前面的快速齐次线性递推(\pageref{LinearRecurrence}页)即可. (快速齐次线性递推的结果是一个序列, 某个给定初值的结果就是点乘, 所以只需要跑一次.)

	如果要求$f_m$, 并且矩阵有$k$个非零项的话, 复杂度就是$O(nk + n\log m\log n)$. (因为暴力求前$2n-1$个$f_i$需要$O(nk)$时间.)

\subsubsection{求矩阵最小多项式}

	矩阵$A$的最小多项式是次数最小的并且$f(A) = 0$的多项式$f$.

	实际上最小多项式就是$\{A^i\}$的最小递推式, 所以直接调用Berlekamp-Massey就好了, 并且显然它的次数不超过$n$.

	瓶颈在于求出$A^i$, 实际上我们只要处理$A^i v$就行了, 每次对向量做递推.
	
	假设$A$有$k$个非零项, 则复杂度为$O(kn + n^2)$.

\subsubsection{求稀疏矩阵的行列式}

	如果能求出特征多项式, 则常数项乘上$(-1)^n$就是行列式, 但是最小多项式不一定就是特征多项式.

	把$A$乘上一个随机对角阵$B$(实际上就是每行分别乘一个随机数), 则$AB$的最小多项式有很大概率就是特征多项式, 最后再除掉$\text{det}\;B$就行了.

	设$A$有$k$个非零项, 则复杂度为$O(kn + n ^ 2)$.

\subsubsection{求稀疏矩阵的秩}

	设$A$是一个$n\times m$的矩阵, 首先随机一个$n\times n$的对角阵$P$和一个$m\times m$的对角阵$Q$, 然后计算$Q A P A^T Q$ 的最小多项式即可.

	实际上不用计算这个矩阵, 因为求最小多项式时要用它乘一个向量, 我们依次把这几个矩阵乘到向量里就行了. 答案就是最小多项式除掉所有$x$因子后剩下的次数.

	设$A$有$k$个非零项, 复杂度为$O(kn + n ^ 2)$.

\subsubsection{解稀疏方程组}

	\paragraph{问题} $Ax = b$, 其中$A$是一个$n \times n$的\textbf{满秩}稀疏矩阵, $b$和$x$是$1\times n$的\textbf{列}向量, $A, b$已知, 需要解出$x$.

	\paragraph{做法} 显然$x = A^{-1} b$. 如果我们能求出$\{A^i b\}$($i \ge 0$)的最小递推式$\{r_{0 \textasciitilde m - 1}\}$($m \le n$), 那么就有结论

	$$ A^{-1} b = -\frac 1 {r_{m - 1}} \sum_{i = 0} ^ {m - 2} A^i b r_{m - 2 - i} $$

	因为$A$是稀疏矩阵, 直接按定义递推出$b \textasciitilde A^{2n - 1} b$即可. 设$A$中有$k$个非零项, 则复杂度为$O(kn + n^2)$.
	
	\inputminted{cpp}{../src/math/解稀疏方程组.cpp}