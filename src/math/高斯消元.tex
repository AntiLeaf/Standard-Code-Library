\paragraph{高斯-约当消元法 Gauss-Jordan}
每次选取当前行绝对值最大的数作为代表元,在做浮点数消元时可以很好地保证精度。

\inputminted{cpp}{../src/math/gauss_jordan.cpp}

\paragraph{解线性方程组}
在矩阵的右边加上一列表示系数即可,如果消成上三角的话最后要倒序回代。

\paragraph{求逆矩阵}
维护一个矩阵 $B$,初始设为 $n$ 阶单位矩阵。在消元的同时对 $B$ 进行一样的操作,那么把 $A$ 消成单位矩阵时 $B$ 就是逆矩阵。

\paragraph{行列式}
消成对角之后把代表元乘起来。如果是任意模数,要注意消元时每交换一次行列要取反一次。
