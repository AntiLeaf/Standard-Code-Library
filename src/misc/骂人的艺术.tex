
\hspace{2em}古今中外没有一个不骂人的人。骂人就是有道德观念的意思,因为在骂人的时候,至少在骂人者自己总觉得那人有该骂的地方。何者该骂,何者不该骂,这个抉择的标准,是极道德的。所以根本不骂人,大可不必。骂人是一种发泄感情的方法,尤其是那一种怨怒的感情。想骂人的时候而不骂,时常在身体上弄出毛病,所以想骂人时,骂骂何妨?

\hspace{2em}但是,骂人是一种高深的学问,不是人人都可以随便试的。有因为骂人挨嘴巴的,有因为骂人吃官司的,有因为骂人反被人骂的,这都是不会骂人的原故。今以研究所得,公诸同好,或可为骂人时之一助乎?

\begin{enumerate}

\item 知己知彼

\hspace{1.8em}骂人是和动手打架一样的,你如其敢打人一拳,你先要自己忖度下,你吃得起别人的一拳否。这叫做知己知彼。骂人也是一样。譬如你骂他是“屈死”,你先要反省,自己和“屈死”有无分别。你骂别人荒唐,你自己想想曾否吃喝嫖赌。否则别人回敬你一二句,你就受不了。所以别人有着某种短处,而足下也正有同病,那么你在骂他的时候只得割爱。

\item 无骂不如己者

\hspace{1.8em}要骂人须要挑比你大一点的人物,比你漂亮一点的或者比你坏得万倍而比你得势的人物,总之,你要骂人,那人无论在好的一方面或坏的一方面都要能胜过你,你才不吃亏。你骂大人物,就怕他不理你,他一回骂,你就算骂着了。因为身份相同的人才肯对骂。在坏的一方面胜过你的,你骂他就如教训一般,他既便回骂,一般人仍不会理会他的。假如你骂一个无关痛痒的人,你越骂他他越得意,时常可以把一个无名小卒骂出名了,你看冤与不冤?

\item 适可而止

\hspace{1.8em}骂大人物骂到他回骂的时候,便不可再骂;再骂则一般人对你必无同情,以为你是无理取闹。骂小人物骂到他不能回骂的时候,便不可再骂;再骂下去则一般人对你也必无同情,以为你是欺负弱者。

\item 旁敲侧击

\hspace{1.8em}他偷东西,你骂他是贼;他抢东西,你骂他是盗,这是笨伯。骂人必须先明虚实掩映之法,须要烘托旁衬,旁敲侧击,于要紧处只一语便得,所谓杀人于咽喉处着刀。越要骂他你越要原谅他,即便说些恭维话亦不为过,这样的骂法才能显得你所骂的句句是真实确凿,让旁人看起来也可见得你的度量。

\item 态度镇定

\hspace{1.8em}骂人最忌浮躁。一语不合,面红筋跳,暴躁如雷,此灌夫骂座,泼妇骂街之术,不足以言骂人。善骂者必须态度镇静,行若无事。普通一般骂人,谁的声音高便算谁占理,谁的来势猛便算谁骂赢,惟真善骂人者,乃能避其锋而击其懈。你等他骂得疲倦的时候,你只消轻轻的回敬他一句,让他再狂吼一阵。在他暴躁不堪的时候,你不妨对他冷笑几声,包管你不费力气,把他气得死去活来,骂得他针针见血。

\item 出言典雅

\hspace{1.8em}骂人要骂得微妙含蓄,你骂他一句要使他不甚觉得是骂,等到想过一遍才慢慢觉悟这句话不是好话,让他笑着的面孔由白而红,由红而紫,由紫而灰,这才是骂人的上乘。欲达到此种目的,深刻之用意固不可少,而典雅之言词则尤为重要。言词典雅可使听者不致刺耳。如要骂人骂得典雅,则首先要在骂时万万别提起女人身上的某一部分,万万不要涉及生理学范围。骂人一骂到生理学范围以内,底下再有什么话都不好说了。譬如你骂某甲,千万别提起他的令堂令妹。因为那样一来,便无是非可言,并且你自己也不免有令堂令妹,他若回敬起来,岂非势均力敌,半斤八两?再者骂人的时候,最好不要加人以种种难堪的名词,称呼起来总要客气,即使他是极卑鄙的小人,你也不妨称他先生,越客气,越骂得有力量。骂得时节最好引用他自己的词句,这不但可以使他难堪,还可以减轻他对你骂的力量。俗话少用,因为俗话一览无遗,不若典雅古文曲折含蓄。

\item 以退为进

\hspace{1.8em}两人对骂,而自己亦有理屈之处,则处于开骂伊始,特宜注意,最好是毅然将自己理屈之处完全承认下来,即使道歉认错均不妨事。先把自己理屈之处轻轻遮掩过去,然后你再重整旗鼓,着着逼人,方可无后顾之忧。即使自己没有理屈的地方,也绝不可自行夸张,务必要谦逊不遑,把自己的位置降到一个不可再降的位置,然后骂起人来,自有一种公正光明的态度。否则你骂他一两句,他便以你个人的事反唇相讥,一场对骂,会变成两人私下口角,是非曲直,无从判断。所以骂人者自己要低声下气,此所谓以退为进。

\item 预设埋伏

\hspace{1.8em}你把这句话骂过去,你便要想想看,他将用什么话骂回来。有眼光的骂人者,便处处留神,或是先将他要骂你的话替他说出来,或是预先安设埋伏,令他骂回来的话失去效力。他骂你的话,你替他说出来,这便等于缴了他的械一般。预设埋伏,便是在要攻击你的地方,你先轻轻的安下话根,然后他骂过来就等于枪弹打在沙包上,不能中伤。

\item 小题大做

\hspace{1.8em}如对方有该骂之处,而题目身小,不值一骂,或你所知不多,不足一骂,那时节你便可用小题大做的方法,来扩大题目。先用诚恳而怀疑的态度引申对方的意思,由不紧要之点引到大题目上去,处处用严谨的逻辑逼他说出不逻辑的话来,或是逼他说出合于逻辑但不合乎理的话来,然后你再大举骂他,骂到体无完肤为止,而原来惹动你的小题目,轻轻一提便了。

\item 远交近攻

\hspace{1.8em}一个时候,只能骂一个人,或一种人,或一派人。决不宜多树敌。所以骂人的时候,万勿连累旁人,即使必须牵涉多人,你也要表示好意,否则回骂之声纷至沓来,使你无从应付。

\end{enumerate}

\hspace{2em}骂人的艺术,一时所能想起来的有上面十条,信手拈来,并无条理。我做此文的用意,是助人骂人。同时也是想把骂人的技术揭破一点,供爱骂人者参考。挨骂的人看看,骂人的心理原来是这样的,也算是揭破一张黑幕给你瞧瞧!
