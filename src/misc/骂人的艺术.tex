
古今中外没有一个不骂人的人. 骂人就是有道德观念的意思, 因为在骂人的时候, 至少在骂人者自己总觉得那人有该骂的地方. 何者该骂, 何者不该骂, 这个抉择的标准, 是极道德的. 所以根本不骂人, 大可不必. 骂人是一种发泄感情的方法, 尤其是那一种怨怒的感情. 想骂人的时候而不骂, 时常在身体上弄出毛病, 所以想骂人时, 骂骂何妨? 

但是, 骂人是一种高深的学问, 不是人人都可以随便试的. 有因为骂人挨嘴巴的, 有因为骂人吃官司的, 有因为骂人反被人骂的, 这都是不会骂人的原故. 今以研究所得, 公诸同好, 或可为骂人时之一助乎? 

\begin{enumerate}

\item 知己知彼

骂人是和动手打架一样的, 你如其敢打人一拳, 你先要自己忖度下, 你吃得起别人的一拳否. 这叫做知己知彼. 骂人也是一样. 譬如你骂他是“屈死”, 你先要反省, 自己和“屈死”有无分别. 你骂别人荒唐, 你自己想想曾否吃喝嫖赌. 否则别人回敬你一二句, 你就受不了. 所以别人有着某种短处, 而足下也正有同病, 那么你在骂他的时候只得割爱. 

\item 无骂不如己者

要骂人须要挑比你大一点的人物, 比你漂亮一点的或者比你坏得万倍而比你得势的人物, 总之, 你要骂人, 那人无论在好的一方面或坏的一方面都要能胜过你, 你才不吃亏. 你骂大人物, 就怕他不理你, 他一回骂, 你就算骂着了. 因为身份相同的人才肯对骂. 在坏的一方面胜过你的, 你骂他就如教训一般, 他既便回骂, 一般人仍不会理会他的. 假如你骂一个无关痛痒的人, 你越骂他他越得意, 时常可以把一个无名小卒骂出名了, 你看冤与不冤? 

\item 适可而止

骂大人物骂到他回骂的时候, 便不可再骂;再骂则一般人对你必无同情, 以为你是无理取闹. 骂小人物骂到他不能回骂的时候, 便不可再骂;再骂下去则一般人对你也必无同情, 以为你是欺负弱者. 

\item 旁敲侧击

他偷东西, 你骂他是贼;他抢东西, 你骂他是盗, 这是笨伯. 骂人必须先明虚实掩映之法, 须要烘托旁衬, 旁敲侧击, 于要紧处只一语便得, 所谓杀人于咽喉处着刀. 越要骂他你越要原谅他, 即便说些恭维话亦不为过, 这样的骂法才能显得你所骂的句句是真实确凿, 让旁人看起来也可见得你的度量. 

\item 态度镇定

骂人最忌浮躁. 一语不合, 面红筋跳, 暴躁如雷, 此灌夫骂座, 泼妇骂街之术, 不足以言骂人. 善骂者必须态度镇静, 行若无事. 普通一般骂人, 谁的声音高便算谁占理, 谁的来势猛便算谁骂赢, 惟真善骂人者, 乃能避其锋而击其懈. 你等他骂得疲倦的时候, 你只消轻轻的回敬他一句, 让他再狂吼一阵. 在他暴躁不堪的时候, 你不妨对他冷笑几声, 包管你不费力气, 把他气得死去活来, 骂得他针针见血. 

\item 出言典雅

骂人要骂得微妙含蓄, 你骂他一句要使他不甚觉得是骂, 等到想过一遍才慢慢觉悟这句话不是好话, 让他笑着的面孔由白而红, 由红而紫, 由紫而灰, 这才是骂人的上乘. 欲达到此种目的, 深刻之用意固不可少, 而典雅之言词则尤为重要. 言词典雅可使听者不致刺耳. 如要骂人骂得典雅, 则首先要在骂时万万别提起女人身上的某一部分, 万万不要涉及生理学范围. 骂人一骂到生理学范围以内, 底下再有什么话都不好说了. 譬如你骂某甲, 千万别提起他的令堂令妹. 因为那样一来, 便无是非可言, 并且你自己也不免有令堂令妹, 他若回敬起来, 岂非势均力敌, 半斤八两? 再者骂人的时候, 最好不要加人以种种难堪的名词, 称呼起来总要客气, 即使他是极卑鄙的小人, 你也不妨称他先生, 越客气, 越骂得有力量. 骂得时节最好引用他自己的词句, 这不但可以使他难堪, 还可以减轻他对你骂的力量. 俗话少用, 因为俗话一览无遗, 不若典雅古文曲折含蓄. 

\item 以退为进

两人对骂, 而自己亦有理屈之处, 则处于开骂伊始, 特宜注意, 最好是毅然将自己理屈之处完全承认下来, 即使道歉认错均不妨事. 先把自己理屈之处轻轻遮掩过去, 然后你再重整旗鼓, 着着逼人, 方可无后顾之忧. 即使自己没有理屈的地方, 也绝不可自行夸张, 务必要谦逊不遑, 把自己的位置降到一个不可再降的位置, 然后骂起人来, 自有一种公正光明的态度. 否则你骂他一两句, 他便以你个人的事反唇相讥, 一场对骂, 会变成两人私下口角, 是非曲直, 无从判断. 所以骂人者自己要低声下气, 此所谓以退为进. 

\item 预设埋伏

你把这句话骂过去, 你便要想想看, 他将用什么话骂回来. 有眼光的骂人者, 便处处留神, 或是先将他要骂你的话替他说出来, 或是预先安设埋伏, 令他骂回来的话失去效力. 他骂你的话, 你替他说出来, 这便等于缴了他的械一般. 预设埋伏, 便是在要攻击你的地方, 你先轻轻的安下话根, 然后他骂过来就等于枪弹打在沙包上, 不能中伤. 

\item 小题大做

如对方有该骂之处, 而题目身小, 不值一骂, 或你所知不多, 不足一骂, 那时节你便可用小题大做的方法, 来扩大题目. 先用诚恳而怀疑的态度引申对方的意思, 由不紧要之点引到大题目上去, 处处用严谨的逻辑逼他说出不逻辑的话来, 或是逼他说出合于逻辑但不合乎理的话来, 然后你再大举骂他, 骂到体无完肤为止, 而原来惹动你的小题目, 轻轻一提便了. 

\item 远交近攻

一个时候, 只能骂一个人, 或一种人, 或一派人. 决不宜多树敌. 所以骂人的时候, 万勿连累旁人, 即使必须牵涉多人, 你也要表示好意, 否则回骂之声纷至沓来, 使你无从应付. 

\end{enumerate}

骂人的艺术, 一时所能想起来的有上面十条, 信手拈来, 并无条理. 我做此文的用意, 是助人骂人. 同时也是想把骂人的技术揭破一点, 供爱骂人者参考. 挨骂的人看看, 骂人的心理原来是这样的, 也算是揭破一张黑幕给你瞧瞧!
