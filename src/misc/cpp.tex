\subsubsection[cmath]{\texttt{cmath}}

\begin{itemize}
	\item \mintinline{cpp}{std::log1p(x)}:(\textbf{注意是数字 1})返回 $\ln(1 + x)$ 的值,$x$ 非常接近 $0$ 时比直接 exp 精确得多。
	\item \mintinline{cpp}{std::hypot(x, y[, z])}:返回平方和的平方根,或者说到原点的欧几里德距离。
\end{itemize}

\subsubsection[algorithm]{\texttt{algorithm}}

\begin{itemize}
	\item \mintinline{cpp}{std::all_of(begin, end, f)}:检查范围内元素调用函数 \mintinline{cpp}{f} 后是否全返回真。类似地还有 \mintinline{cpp}{std::any_of} 和 \mintinline{cpp}{std::none_of}。
	\item \mintinline{cpp}{std::for_each(begin, end, f)}:对范围内所有元素调用一次 \mintinline{cpp}{f}。如果传入的是引用,也可以用 \mintinline{cpp}{f} 修改。(例如 \mintinline{cpp}{for_each(a, a + n, [](int &x){ cout << ++x << "\n"; })})
	\item \mintinline{cpp}{std::for_each_n(begin, n, f)}:同上,只不过范围改成了从 \mintinline{cpp}{begin} 开始的 n 个元素。
	\item \mintinline{cpp}{std::copy(), std::copy_n()}:用法谁都会,但标准里说如果元素是可平凡复制的(比如 \mintinline{cpp}{int}),那么它会避免批量赋值,并且调用 \mintinline{cpp}{std::memmove()} 之类的快速复制函数。(一句话总结:它跑得快)
	\item \mintinline{cpp}{std::rotate(begin, mid, end)}:\textbf{向前} 循环移动,移动后 \mintinline{cpp}{mid} 位置的元素会跑到 \mintinline{cpp}{begin} 位置。C++11 起会返回 \mintinline{cpp}{begin} 位置的元素移动后的位置。
	\item \mintinline{cpp}{std::unique(begin, end)}:去重,返回去重后的 \mintinline{cpp}{end}。
	\item \mintinline{cpp}{std::partition(begin, end, f)}:把 \mintinline{cpp}{f} 为 \mintinline{cpp}{true} 的放在前面,\mintinline{cpp}{false} 的放在后面,返回值是第二部分的开头,\textbf{不保持相对顺序}。如果要保留相对顺序可以用 \mintinline{cpp}{std::stable_partition()},比如写整体二分。
	\item \mintinline{cpp}{std::partition_copy(begin, end, begin_t, begin_f, f)}:不修改原数组,把 \mintinline{cpp}{true} 的扔到 \mintinline{cpp}{begin_t},\mintinline{cpp}{false} 的扔到 \mintinline{cpp}{begin_f}。返回值是两部分结尾的迭代器的 \mintinline{cpp}{pair}。
	\item \mintinline{cpp}{std::equal_range(begin, end, x)}:在已经排好序的数组里找到等于 \mintinline{cpp}{x} 的范围。
	\item \mintinline{cpp}{std::minmax(a, b)}:返回 \mintinline{cpp}{pair(min(a, b), max(a, b))}。但是要注意 \textbf{返回的是引用},所以不能直接用来交换 $l, r$。
\end{itemize}

\subsubsection[std::tuple]{\texttt{std::tuple}}

\begin{itemize}
	% \item \mintinline{cpp}{std::make_tuple(...)}:返回构造好的 \mintinline{cpp}{tuple}。
	\item \mintinline{cpp}{std::get<i>(tup)}:返回 \mintinline{cpp}{tuple} 的第 \mintinline{cpp}{i} 项。
	\item \mintinline{cpp}{std::tuple_cat(...)}:传入几个 \mintinline{cpp}{tuple},返回按顺序连起来的 \mintinline{cpp}{tuple}。
	\item \mintinline{cpp}{std::tie(x, y, z, ...)}:把传入的变量的 \textbf{左值引用} 绑起来作为 \mintinline{cpp}{tuple} 返回,例如可以 \mintinline{cpp}{std::tie(x, y, z) = std::tuple(a, b, c)}。
\end{itemize}

\subsubsection[complex]{\texttt{complex}}

\begin{itemize}
	\item \mintinline{cpp}{complex<double> imaginary = 1i, x = 2 + 3i}:语法糖,可以这样直接构造复数。
	\item \mintinline{cpp}{real/imag(x)}:返回实部/虚部。
	\item \mintinline{cpp}{conj(x)}:返回共轭复数。
	\item \mintinline{cpp}{arg(x)}:返回辐角。
	\item \mintinline{cpp}{norm(x)}:返回模的平方。(直接求模用 \mintinline{cpp}{abs(x)} 就行。)
	\item \mintinline{cpp}{polar(len, theta)}:用模长和辐角构造复数。
\end{itemize}
