\subsubsection{vector}
	\begin{itemize}
		\item \mintinline{cpp}{vector(int nSize)}: 创建一个vector, 元素个数为nSize
		\item \mintinline{cpp}{vector(int nSize, const T &value)}: 创建一个vector, 元素个数为nSize, 且值均为value
		\item \mintinline{cpp}{vector(begin, end)}: 复制[begin, end)区间内另一个数组的元素到vector中
		\item \mintinline{cpp}{void assign(int n, const T &x)}: 设置向量中前n个元素的值为x
		\item \mintinline{cpp}{void assign(const_iterator first, const_iterator last)}: 向量中[first, last)中元素设置成当前向量元素
		\item \mintinline{cpp}{void emplace_back(Args&&... args)}: 自动构造并push\_back一个元素, 例如对一个存储pair的vector可以 \mintinline{cpp}{v.emplace_back(x, y)}
	\end{itemize}

\subsubsection{list}
	\begin{itemize}
		\item \mintinline{cpp}{assign()} 给list赋值 
		\item \mintinline{cpp}{back()} 返回最后一个元素 
		\item \mintinline{cpp}{begin()} 返回指向第一个元素的迭代器 
		\item \mintinline{cpp}{clear()} 删除所有元素 
		\item \mintinline{cpp}{empty()} 如果list是空的则返回true 
		\item \mintinline{cpp}{end()} 返回末尾的迭代器
		\item \mintinline{cpp}{erase()} 删除一个元素
		\item \mintinline{cpp}{front()} 返回第一个元素
		\item \mintinline{cpp}{insert()} 插入一个元素到list中
		\item \mintinline{cpp}{max_size()} 返回list能容纳的最大元素数量
		\item \mintinline{cpp}{merge()} 合并两个list
		\item \mintinline{cpp}{pop_back()} 删除最后一个元素
		\item \mintinline{cpp}{pop_front()} 删除第一个元素
		\item \mintinline{cpp}{push_back()} 在list的末尾添加一个元素
		\item \mintinline{cpp}{push_front()} 在list的头部添加一个元素
		\item \mintinline{cpp}{rbegin()} 返回指向第一个元素的逆向迭代器
		\item \mintinline{cpp}{remove()} 从list删除元素
		\item \mintinline{cpp}{remove_if()} 按指定条件删除元素
		\item \mintinline{cpp}{rend()} 指向list末尾的逆向迭代器
		\item \mintinline{cpp}{resize()} 改变list的大小
		\item \mintinline{cpp}{reverse()} 把list的元素倒转
		\item \mintinline{cpp}{size()} 返回list中的元素个数
		\item \mintinline{cpp}{sort()} 给list排序
		\item \mintinline{cpp}{splice()} 合并两个list
		\item \mintinline{cpp}{swap()} 交换两个list
		\item \mintinline{cpp}{unique()} 删除list中重复的元
	\end{itemize}

\subsubsection{unordered\_set / unordered\_map}

\begin{itemize}
	\item \mintinline{cpp}{unordered_map<int, int, hash>}: 自定义哈希函数, 其中hash是一个带重载括号的类.
\end{itemize}