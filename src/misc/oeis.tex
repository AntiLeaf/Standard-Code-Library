\begin{enumerate}

\item \textbf{卡特兰数(A000108)}

1, 1, 2, 5, 14, 42, 132, 429, 1430, 4862, 16796, 58786, 208012, 742900, 2674440, 9694845, 35357670, \dots

性质见``数学''部分.

\item \textbf{Lucas数(A000032)}

2, 1, 3, 4, 7, 11, 18, 29, 47, 76, 123, 199, 322, 521, 843, 1364, 2207, 3571, 5778, 9349, 15127, \dots

\item \textbf{斐波那契数(A000045)}

0, 1, 1, 2, 3, 5, 8, 13, 21, 34, 55, 89, 144, 233, 377, 610, 987, 1597, 2584, 4181, 6765, 10946, \dots

\item \textbf{泰波那契数(Tribonacci, A000071)}

0, 0, 1, 1, 2, 4, 7, 13, 24, 44, 81, 149, 274, 504, 927, 1705, 3136, 5768, 10609, 19513, 35890, \dots

$ a_0 = a_1 = 0,\; a_2 = 1,\; a_n = a_{n - 1} + a_{n - 2} + a_{n - 3} $.

\item \textbf{Pell数(A0000129)}

0, 1, 2, 5, 12, 29, 70, 169, 408, 985, 2378, 5741, 13860, 33461, 80782, 195025, 470832, 1136689, \dots

$ a_0 = 0,\; a_1 = 1,\; a_n = 2a_{n - 1} + a_{n - 2} $.

\item \textbf{帕多万(Padovan)数(A0000931)}

1, 0, 0, 1, 0, 1, 1, 1, 2, 2, 3, 4, 5, 7, 9, 12, 16, 21, 28, 37, 49, 65, 86, 114, 151, 200, 265, 351, 465, 616, 816, 1081, 1432, 1897, 2513, 3329, 4410, 5842, 7739, 10252, 13581, 17991, 23833, 31572, \dots

$a_0 = 1,\; a_1 = a_2 = 0,\; a_n = a_{n - 2} + a_{n - 3}$.

\item \textbf{划分数(A000041)}

1, 1, 2, 3, 5, 7, 11, 15, 22, 30, 42, 56, 77, 101, 135, 176, 231, 297, 385, 490, 627, 792, 1002, \dots

\item \textbf{贝尔数(A000110)}

1, 1, 2, 5, 15, 52, 203, 877, 4140, 21147, 115975, 678570, 4213597, 27644437, 190899322, 1382958545, \dots

\item \textbf{错位排列数(A0000166)}

1, 0, 1, 2, 9, 44, 265, 1854, 14833, 133496, 1334961, 14684570, 176214841, 2290792932, 32071101049, \dots

\item \textbf{默慈金数(Motzkin numbers, A001006)}

1, 1, 2, 4, 9, 21, 51, 127, 323, 835, 2188, 5798, 15511, 41835, 113634, 310572, 853467, 2356779, \dots

$ M_{n + 1} = M_n + \sum_{i = 0} ^ {n - 1} M_i M_{n - 1 - i} = \frac {(2n + 3)M_n + 3n M_{n - 1}} {n + 3} $

$ M_n = \sum_{i = 0} ^ {\left\lfloor \frac n 2 \right\rfloor} {n \choose 2i} Catalan_{i} $

在圆上的$n$个\textbf{不同的}点之间画任意条不相交的弦的方案数.

也等价于在网格图上, 每次可以走右上, 右下, 正右方一步, 且不能走到$y<0$的位置, 在此前提下从$(0, 0)$走到$(n, 0)$的方案数.



\item \textbf{四个柱子的汉诺塔(A007664)}

0, 1, 3, 5, 9, 13, 17, 25, 33, 41, 49, 65, 81, 97, 113, 129, 161, 193, 225, 257, 289, 321, 385, 449, \dots

差分之后可以发现其实就是1次+1, 2次+2, 3次+4, 4次+8\dots 的规律.

\item \textbf{梅森质数(A000668)}

3, 7, 31, 127, 8191, 131071, 524287, 2147483647, 2305843009213693951, 618970019642690137449562111, 162259276829213363391578010288127,\\170141183460469231731687303715884105727

$p$是质数, 同时$2^p - 1$也是质数.

\end{enumerate}