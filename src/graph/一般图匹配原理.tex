设图 $G$ 的 Tutte 矩阵是 $\tilde A$,首先是最基础的引理:

\begin{itemize}
	\item $G$ 的最大匹配大小是 $\frac 1 2 \text{rank}{\tilde A}$;
	
	\item $\left({\tilde A} ^{-1}\right) _{i, j} \ne 0 \iff G-\{v_i, v_j\}$ 有完美匹配。
		\subitem (考虑到逆矩阵与伴随矩阵的关系,这是显然的。)
\end{itemize}

构造最大匹配的方法见板子。对于更一般的问题,可以借助构造方法转化为完美匹配问题。

设最大匹配的大小为 $k$,新建 $n - 2k$ 个辅助点,让它们和其他所有点连边,那么如果一个点匹配了一个辅助点,就说明它在原图的匹配中不匹配任何点。

\paragraph{可行边} 对原图中的任意一条边 $(u, v)$,如果删掉 $u, v$ 后新图仍然有完美匹配(即 ${\tilde A} ^ {-1}_{u, v} \ne 0$),则它是一条可行边。

\paragraph{必须边} \textbf{待补充}。

\paragraph{必须点} 可以删掉这个点和一个辅助点,然后判断剩下的图是否还有完美匹配,如果有则说明它不是必须的,否则是必须的。只需要用到逆矩阵即可。

\paragraph{可行点} 显然对于任意一个点,只要它不是孤立点,就是可行点。
