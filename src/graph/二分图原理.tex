\begin{itemize}

\item \textbf{最大匹配的可行边与必须边、关键点}

以下的“残量网络”指网络流图的残量网络。

\paragraph{可行边} 一条边的两个端点在残量网络中处于同一个 SCC,不论是正向边还是反向边。

\paragraph{必须边} 一条属于当前最大匹配的边,且残量网络中两个端点不在同一个 SCC 中。

\paragraph{关键点/必须点} 这里不考虑网络流图而只考虑原始的图,将匹配边改成从右到左之后从左边的每个未匹配点进行 floodfill,左边没有被标记的点即为关键点。右边同理。

\item \textbf{独立集}

二分图独立集可以看成最小割问题,割掉最少的点使得 S 和 T 不连通,则剩下的点自然都在独立集中。

所以独立集输出方案就是求出不在最小割中的点,独立集的必须点/可行点就是最小割的不可行点/非必须点。

割点等价于割掉它与源点或汇点相连的边,可以通过设置中间的边权为无穷以保证不能割掉中间的边,然后按照上面的方法判断即可。

(由于一个点最多流出一个流量,所以中间的边权其实是可以任取的。)

\item \textbf{二分图最大权匹配}

二分图最大权匹配的对偶问题是 \textbf{最小顶标和}\ 问题,即:为图中的每个顶点赋予一个非负顶标,使得对于任意一条边,两端点的顶标和都要不小于边权,最小化顶标之和。

KM 算法的原理实际上就是求最小顶标和。

\end{itemize}
