也叫原始-对偶费用流。

原理和求多源最短路的 Johnson 算法是一样的,都是给每个点维护一个势 $h_u$,使得对任何有向边 $u\to v$ 都满足 $w + h_u - h_v \ge 0$。

如果有负费用则从 $s$ 开始跑一遍 SPFA 初始化,否则可以直接初始化 $h_u = 0$。

每次增广时得到的路径长度就是 $d_{s, t} + h_t$,增广之后让所有 $h_u = h'_u + d'_{s, u}$,直到 $d_{s, t} = +\infty$(最小费用最大流)或 $d_{s, t} \ge 0$(最小费用流)为止。

注意最大费用流要转成取负之后的最小费用流,因为 Dijkstra 求的是最短路。

\inputminted{cpp}{../src/graph/dijkstra费用流.cpp}
