
\noindent{
	{\large\bfseries{通用}}
	\begin{itemize}
		\item 出现次数大于$\sqrt n$的数不会超过$\sqrt n$个
		\item 对于带修改问题, 如果不方便分治或者二进制分组, 可以考虑对操作分块, 每次查询时暴力最后的$\sqrt n$个修改并更正答案
		\item {\bfseries 根号分治}: 如果分治时每个子问题需要$O(N)$(N是全局问题的大小)的时间, 而规模较小的子问题可以$O(n^2)$解决, 则可以使用根号分治
			\begin{itemize}
				\item 规模大于$\sqrt n$的子问题用$O(N)$的方法解决, 规模小于$\sqrt n$的子问题用$O(n^2)$暴力
				\item 规模大于$\sqrt n$的子问题最多只有$\sqrt n$个
				\item 规模不大于$\sqrt n$的子问题大小的平方和也必定不会超过$n\sqrt n$
			\end{itemize}
		\item 如果输入规模之和不大于$n$(例如给定多个小字符串与大字符串进行询问), 那么规模超过$\sqrt n$的问题最多只有$\sqrt n$个
	\end{itemize}
}

\noindent{
	{\large\bfseries{序列}}
	\begin{itemize}
		\item 某些维护序列的问题可以用分块/块状链表维护
		\item 对于静态区间询问问题, 如果可以快速将左/右端点移动一位, 可以考虑莫队
			\begin{itemize}
				\item 如果强制在线可以分块预处理, 但是一般空间需要$n\sqrt n$ 
					\begin{itemize}
						\item 例题: 询问区间中有几种数出现次数恰好为$k$, 强制在线
					\end{itemize}
				\item 如果带修改可以试着想一想带修莫队, 但是复杂度高达$n^{\frac 5 3}$
			\end{itemize}
		\item 线段树可以解决的问题也可以用分块来做到$O(1)$询问或是$O(1)$修改, 具体要看哪种操作更多
	\end{itemize}
}

\noindent{
	{\large\bfseries{树}}
	\begin{itemize}
		\item 与序列类似, 树上也有树分块和树上莫队
			\begin{itemize}
				\item 树上带修莫队很麻烦, 常数也大, 最好不要先考虑
				\item 树分块不要想当然
			\end{itemize}
		\item 树分治也可以套根号分治, 道理是一样的
	\end{itemize}
}

\noindent{
	{\large\bfseries{字符串}}
	\begin{itemize}
		\item 循环节长度大于$\sqrt n$的子串最多只有$O(n)$个, 如果是极长子串则只有$O(\sqrt n)$个
	\end{itemize}
}

\noindent {
	{\large\bfseries{关于莫队}}

	莫队是可以改造成只有插入和撤销(或者只有删除和撤销)的版本的.

	例如维护dfs序时就可以使用链表, 配合只有删除的莫队就可以做到$O(n\sqrt n)$.

	另外如果$n$和$q$不平衡, 块大小应该设为$\frac n {\sqrt q}$.
}