\sout{让fstqwq手撕}

\begin{itemize}
	\item 如果$M = 2^k$, 则只能维护$[1, M - 2]$范围
	\item 找叶子: $i$对应的叶子就是$i + M$
	\item 单点修改: 找到叶子然后向上跳
	\item 区间查询: 左右区间各扩展一位, 转换成开区间查询
\begin{minted}{cpp}
int query(int l, int r) {
	l += M - 1;
	r += M + 1;

	int ans = 0;
	while (l ^ r != 1) {
		ans += sum[l ^ 1] + sum[r ^ 1];

		l >>= 1;
		r >>= 1;
	}

	return ans;
}
\end{minted}

\end{itemize}
	
区间修改要标记永久化,并且求区间和和求最值的代码不太一样 \\

\textbf{区间加, 区间求和}
\begin{minted}{cpp}
void update(int l, int r, int d) {
	int len = 1, cntl = 0, cntr = 0; // cntl、cntr是左右两边分别实际修改的区间长度
	for (l += n - 1, r += n + 1; l ^ r ^ 1; l >>= 1, r >>= 1, len <<= 1) {
		tree[l] += cntl * d, tree[r] += cntr * d;
		if (~l & 1) tree[l ^ 1] += d * len, mark[l ^ 1] += d, cntl += len;
		if (r & 1) tree[r ^ 1] += d * len, mark[r ^ 1] += d, cntr += len;
	}

	for (; l; l >>= 1, r >>= 1)
		tree[l] += cntl * d, tree[r] += cntr * d;
}

int query(int l, int r) {
	int ans = 0, len = 1, cntl = 0, cntr = 0;
	for (l += n - 1, r += n + 1; l ^ r ^ 1; l >>= 1, r >>= 1, len <<= 1) {
		ans += cntl * mark[l] + cntr * mark[r];
		if (~l & 1) ans += tree[l ^ 1], cntl += len;
		if (r & 1) ans += tree[r ^ 1], cntr += len;
	}

	for (; l; l >>= 1, r >>= 1)
		ans += cntl * mark[l] + cntr * mark[r];

	return ans;
}
\end{minted}

\textbf{区间加, 区间求最大值}
\begin{minted}{cpp}
void update(int l, int r, int d) {
	for (l += N - 1, r += N + 1; l ^ r ^ 1; l >>= 1, r >>= 1) {
		if (l < N) {
			tree[l] = max(tree[l << 1], tree[l << 1 | 1]) + mark[l];
			tree[r] = max(tree[r << 1], tree[r << 1 | 1]) + mark[r];
		}

		if (~l & 1) {
			tree[l ^ 1] += d;
			mark[l ^ 1] += d;
		}
		if (r & 1) {
			tree[r ^ 1] += d;
			mark[r ^ 1] += d;
		}
	}

	for (; l; l >>= 1, r >>= 1)
		if (l < N) tree[l] = max(tree[l << 1], tree[l << 1 | 1]) + mark[l],
					tree[r] = max(tree[r << 1], tree[r << 1 | 1]) + mark[r];
}

void query(int l, int r) {
	int maxl = -INF, maxr = -INF;

	for (l += N - 1, r += N + 1; l ^ r ^ 1; l >>= 1, r >>= 1) {
		maxl += mark[l];
		maxr += mark[r];

		if (~l & 1)
			maxl = max(maxl, tree[l ^ 1]);
		if (r & 1)
			maxr = max(maxr, tree[r ^ 1]);
	}

	while (l) {
		maxl += mark[l];
		maxr += mark[r];

		l >>= 1;
		r >>= 1;
	}
	
	return max(maxl, maxr);
}
\end{minted}