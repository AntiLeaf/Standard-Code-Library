KMP 和 AC 自动机的 \texttt{fail} 指针存储的都是它在串或者字典树上的最长后缀,因此要判断两个前缀是否互为后缀时可以直接用 \texttt{fail} 指针判断。当然它不能做最长公共后缀,不过可以用一个树链的并来做子串问题。

后缀数组利用的主要是 LCP 长度可以按照字典序做 RMQ 的性质,与某个串的 LCP 长度 $\ge$ 某个值的后缀形成一个区间。另外一个比较好用的性质是本质不同的子串个数 = 所有子串数 - 字典序相邻的串的 \texttt{height}。

后缀自动机实际上可以接受的是所有后缀, 如果把中间状态也算上的话就是所有子串. 它的 \text{fail} 指针代表的也是当前串的后缀,不过注意每个状态可以代表很多状态,只要右端点在 \texttt{right} 集合中且长度处在 $(val_{par_p}, val_p]$ 中的串都被它代表。

后缀自动机的 \texttt{fail} 树也就是 \textbf{反串} 的后缀树。每个结点代表的串和后缀自动机同理,两个串的 LCP 也就是他们在后缀树上的 LCA。
