\subsubsection{103388A Assigning Prizes 容斥}
	
	\paragraph{题意} 给定一个长为 $n$ 的序列 $a_i$, 要求构造非严格递减序列 $b_i$, 满足 $a_i \le b_i \le R$, 求方案数. $n \le 5 \times 10 ^ 3, R, a_i \le 10 ^ 9$.
	
	\paragraph{做法} $a_i$ 的范围太大了, 不能简单地记录上一位的值.

	考虑使用容斥. 方便起见把 $a_i$ 直接变成 $R - a_i + 1$, 条件就变成了 $b_i \le a_i$ 且 $b_i \ge b_{i - 1}$.

	这里有两个限制条件, 可以固定 $b_i \le a_i$ 是必须满足的条件, 只对 $b_i \ge b_{i - 1}$ 使用容斥, 枚举哪些位置是比上一位小的(违反限制), 其他位置随意.

	枚举后的形态一定是有若干个区间是严格递减的, 其他位置随意. 考虑如果一个区间 $[l, r]$ 是严格递减的, 显然所有的数都 $< a_l$, 所以这段区间的方案数就是 ${a_l \choose r - l + 1}$. 另外实际上 $b_l$ 是没有违反限制的, 所以这里对系数的贡献是 $(-1) ^ {r - l}$.

	考虑令 $dp_i$ 表示只考虑前 $i$ 个位置的答案, 转移时自然就是枚举一个 $j$, 然后计算 $dp_{j - 1}$ 乘上区间 $[j, i]$ 严格递减的方案数. 另外还有一种情况是 $b_i$ 没有违反限制, 这时显然直接在 $dp_{i - 1}$ 的基础上乘上一个 $a_i$ 就好了. (转移时还要注意, 由于枚举的是严格递减区间, 自然就不能枚举只有一个数的区间.)

	\inputminted{cpp}{../src/dp/103388A.cpp}