\documentclass{article}
\title{Standard Code Library}
\author{AntiLeaf}
\date{}

% \usepackage{CTEX}

% \usepackage{CJKpunct}
% \punctstyle{quanjiao}

\usepackage{graphicx, amssymb, amsmath, textcomp, booktabs}
\usepackage[libertine,vvarbb]{newtxmath}
\usepackage[scr=rsfso]{mathalfa}
%\usepackage[lining,semibold,type1]{libertine} % a bit lighter than Times--no osf in math
\usepackage[T1]{fontenc} % best for Western European languages
\usepackage{minted}
\usepackage{listings, color, setspace, titlesec, fancyhdr, mdframed, multicol}
\usepackage{fontspec}
\usepackage{ucharclasses}
\usepackage{xunicode, xltxtra}
\usepackage{pdfpages}
\usepackage{tocloft}
\usepackage{nameref}
\usepackage{verbatim}
\usepackage{relsize}

\usepackage{color,xcolor}

\definecolor{light-gray}{gray}{0.9}    % 1.灰度

\XeTeXlinebreaklocale "zh"
\XeTeXlinebreakskip = 0pt plus 1pt

%configure space between the two columns
\setlength{\columnsep}{13pt}

%configure fonts
\setmonofont{Consolas}[Scale=0.775]
\newfontfamily\substitutefont{等线}[Scale=0.9]
\setTransitionsForChinese{\begingroup\substitutefont}{\endgroup}

%configure minted to display codes
\definecolor{Gray}{rgb}{0.9,0.9,0.9}

%configure section style of table of content
\renewcommand\cftsecfont{\Large}

%configure section style
\titleformat{\section}
{\huge}			% The style of the section title
{\thesection.}				% a prefix
{4pt}						% How much space exists between the prefix and the title
{}					% How the section is represented
% \titleformat{\section}{\huge}{}{0pt}{}
\titlespacing{\section}{0pt}{0pt}{0pt}
\titlespacing{\subsection}{0pt}{0pt}{0pt}
\titlespacing{\subsubsection}{0pt}{0pt}{0pt}

\usepackage{fancyhdr}
\usepackage[inner=1.35cm, outer=0.9cm, top=1.7cm, bottom=1.7cm]{geometry}

\pagestyle{fancy}
\setlength{\headsep}{0.1cm}
\chead{Standard Code Library}
\rhead{\leftmark}
\lhead{Shanghai Jiao Tong University}

\cfoot{\thepage}

\renewcommand{\headrulewidth}{0.5pt}
\renewcommand{\footrulewidth}{0.5pt}

\setminted[cpp]{
	style=solarized-light,
	mathescape,
	linenos,
	autogobble,
	baselinestretch=0.9,
	tabsize=4,
	fontsize=\normalsize,
	%bgcolor=Gray,
	frame=single,
	framesep=1mm,
	framerule=0.3pt,
	numbersep=1mm,
	breaklines=true,
	breaksymbolsepleft=2pt,
	%breaksymbolleft=\raisebox{0.8ex}{ \small\reflectbox{\carriagereturn}}, %not moe!
	%breaksymbolright=\small\carriagereturn,
	breakbytoken=false,
	showtabs=true,
	tab={\relscale{1.08} $\color{light-gray}{\vert} \ \ \ $ \relscale{1}},
}

\setminted[python]{
	style=solarized-light,
	mathescape,
	linenos,
	autogobble,
	baselinestretch=0.9,
	tabsize=4,
	fontsize=\normalsize,
	%bgcolor=Gray,
	frame=single,
	framesep=0.8mm,
	framerule=0.3pt,
	numbersep=0.8mm,
	breaklines=true,
	breaksymbolsepleft=2pt,
	%breaksymbolleft=\raisebox{0.8ex}{ \small\reflectbox{\carriagereturn}}, %not moe!
	%breaksymbolright=\small\carriagereturn,
	breakbytoken=false,
	showtabs=true,
	tab={\relscale{1.08} $\color{light-gray}{\vert} \ \ \ $ \relscale{1}},
}

\begin{document}

	\begin{titlepage}
		\maketitle
		\newpage
	\end{titlepage}

	\begin{multicols}{2}
		
		\begin{spacing}{1}
			\tableofcontents
		\end{spacing}
		\newpage

	\end{multicols}

	\begin{multicols}{2}

		\section{图论}

			\subsection{最小生成树}
				% \subsubsection{Boruvka算法}
					
				\subsubsection{动态最小生成树}
					\inputminted{cpp}{../src/图论/动态最小生成树.cpp}
				
				% \subsubsection{最小树形图(朱刘算法)}

				
				% \subsubsection{斯坦纳树}
			
			
			\subsection{最短路}
				% \subsubsection{Dijkstra}
					
				
				% \subsubsection{差分约束}


				\subsubsection{k短路}
					\inputminted{cpp}{../src/图论/k短路.cpp}

			\subsection{仙人掌}
				\subsubsection{仙人掌DP}
					\inputminted{cpp}{../src/图论/仙人掌DP.cpp}
			
			\subsection{二分图}
				% \subsubsection{匈牙利}


				% \subsubsection{Hopcroft-Karp}


				\subsubsection{KM二分图最大权匹配}
					\inputminted{cpp}{../src/图论/KM二分图最大权匹配.cpp}

				% \subsubsection{二分图原理}
			
			
			\subsection{一般图匹配}
				\subsubsection{高斯消元}
					\inputminted{cpp}{../src/图论/基于线性代数的一般图匹配.cpp}

				\subsubsection{带花树}
					\inputminted{cpp}{../src/图论/带花树.cpp}
			
			% \subsection{支配树}
			

			% \subsection{2-SAT}


			\subsection{最大流}
				\subsubsection{Dinic}
					\inputminted{cpp}{../src/图论/Dinic.cpp}

				\subsubsection{ISAP}
					\inputminted{cpp}{../src/图论/ISAP.cpp}

				\subsubsection{HLPP最高标号预流推进}
					\inputminted{cpp}{../src/图论/HLPP.cpp}
			
			% \subsection{费用流}
				% \subsubsection{SPFA费用流}


				% \subsubsection{Dijkstra费用流}


				% \subsubsection{zkw费用流}
			

			% \subsection{网络流原理}
				% \subsubsection{最小割}


				% \subsubsection{费用流}


				% \subsubsection{常见建图方法}


				% \subsubsection{例题}
			

			% \subsection{弦图}
				% \subsubsection{完美消除序列、最大势算法}


				% \subsubsection{性质}
			

			% \subsection{其他}
				% \subsubsection{Stoer-Wagner全局最小割}


				% \subsubsection{最小割树}


				% \subsubsection{最大团搜索}
			

		\section{字符串}

			\subsection{AC自动机}
				\inputminted{cpp}{../src/字符串/AC自动机.cpp}

			\subsection{后缀数组}
				% \subsubsection{SA-IS}

			
				\subsubsection{SAMSA}
					\inputminted{cpp}{../src/字符串/SAMSA.cpp}

			% \subsection{后缀平衡树}


			\subsection{后缀自动机}
				\inputminted{cpp}{../src/字符串/后缀自动机.cpp}

			\subsection{回文树}
				\inputminted{cpp}{../src/字符串/回文树.cpp}

				\subsubsection{广义回文树}
					\inputminted{cpp}{../src/字符串/广义回文树.cpp}


			% \subsection{序列自动机}


			\subsection{Manacher马拉车}
				\inputminted{cpp}{../src/字符串/manacher.cpp}

			\subsection{KMP}
				

				\subsubsection{ex-KMP}
					\inputminted{cpp}{../src/字符串/exKMP.cpp}
			
			% \subsection{All Substring LCS}


			% \subsection{字符串原理}


		\section{数学}

			\subsection{插值}
				\subsubsection{牛顿插值}


				\subsubsection{拉格朗日插值}

			
			\subsection{多项式}
				\subsubsection{FFT}
					\inputminted{cpp}{../src/数学/FFT.cpp}

				\subsubsection{NTT}
					\inputminted{cpp}{../src/数学/NTT.cpp}

				\subsubsection{任意模数卷积(三模数NTT)}
					\inputminted{cpp}{../src/数学/任意模数卷积.cpp}

				% \subsection{毛梯梯}

			
				\subsubsection{多项式操作}
					\inputminted{cpp}{../src/数学/多项式操作.cpp}

				% \subsubsection{多项式多点求值}


				% \subsubsection{多项式快速插值}


				% \subsubsection{多项式复合与复合逆}

				\subsubsection{拉格朗日反演}


				% \subsubsection{分治FFT}


				\subsubsection{半在线卷积}
					\inputminted{cpp}{../src/数学/半在线卷积.cpp}

			\subsection{FWT快速沃尔什变换}
				\inputminted{cpp}{../src/数学/FWT.cpp}

			\subsection{单纯形}
				\inputminted{cpp}{../src/数学/单纯形.cpp}

				% \subsubsection{线性规划对偶原理}


			\subsection{线性代数}
				% \subsubsection{高斯消元}


				% \subsubsection{任意模数高斯消元}


				% \subsubsection{自由元搜索}


				\subsubsection{线性基}
					% \inputminted{cpp}{../src/数学/线性基.cpp}


				% \subsubsection{线性代数知识}


			% \subsection{自适应Simpson积分}


			% \subsection{常见结论}
				% \subsubsection{线性齐次递推求通项}



	\end{multicols}

\end{document}